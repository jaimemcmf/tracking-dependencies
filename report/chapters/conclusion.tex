\chapter{Conclusion}

While developing this process it was possible to acknowledge the complexity of common packages such as \textit{pip} and \textit{numpy}, how this complexity differs from project to project, as well as understanding the dependency search and installation process used by Python's default package manager \textit{pip}.

The objectives of this project were halfway met. On the one hand, the first part regarding dependency search was fully achieved: this project's programs are able to have the expected behavior when finding dependencies of a given package.

On the other hand, the static code analysis was not achieved as expected. This can be partially explained by the lack of access to malicious source code and for the abundant diversity of \textit{setup.py}'s content. Even though there is a standard structure followed by most developers, many have different approaches which results in multiple false positives.

The unsatisfying results could be attenuated by having access to a bigger library of malicious packages and by having more time and knowledge on the Python programming language.